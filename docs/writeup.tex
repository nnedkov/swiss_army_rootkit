\documentclass[10pt, letterpaper]{scrartcl}
\usepackage [english]{babel}
\usepackage{enumitem}
\usepackage{geometry}
\usepackage{color}
\usepackage{tikz}
\usepackage{listings}
\usepackage{latexsym}
\usepackage{amsmath}
\usepackage{multirow}
\usetikzlibrary{snakes}
\usetikzlibrary{patterns}
\usepackage[loose]{subfigure}
\usepackage[pdfborder={0 0 0}]{hyperref}


\geometry{margin=2.5cm}

\title{Our\_rootkit\_name - Rootkit Team 105}
\subtitle{TUM \\Chair of IT Security\\  Rootkit Programming WS2015/16}
\author{Nedko Stefanov Nedkov \and Matei Pavaluca}
\date{\today}

\begin{document}
\maketitle
\tableofcontents
\newpage

\section{TODO: Introduction}
This rootkit has been implemented as part of the "Rootkit Programming" praktikum
course at the TUM in WS2015/16. It offers basic functionality such keylogging,
file/process/socket/packet masking, privilege escalation, port knocking and basic self stealthness but also provides more
advanced features such as command and control over a stealth tcp channel and
even a shell backdoor.

Chapter \ref{sec:compiling_and_installing} offers an overview of the machine
used for deploying this rootkit and describes how it can be properly built,
inserted, controled and unloadded.
Chapter \ref{sec:implementation} discusses the implementations of the different
functionalities while chapter \ref{sec:vulnerabilities} points to a few
vulnerabilities that can be exploited in order to detect and remove this
rootkit..

\section{Compiling and Installing}\label{sec:compiling_and_installing}
\subsection{System configuration}
This rootkit has been implemented and tested on a specific system. It should
behave well on other configurations, but this is the only tested one:
\begin{itemize}
    \item VirtualBox VM (5GB HDD, 1024MB memory) emulating a single x86-amd64 CPU
    \item Debian Jessie 8.2 x86-64
    \item Linux Kernel 4.1.10 x86-64 (Vanilla) built using the configuration of the Debian kernel as a base
\end{itemize}

\subsection{Building}
Building requires the installed kernel sources and the source code of the
rootkit. To build, simply enter the source folder and \texttt{make}.


This will compile the rootkit with all its functionality. If the build was
successful the file \texttt{rootkit.ko} as well as other object files will be created in the source folder.

\subsection{Loading}
The rootkit can be inserted by issuing the command \texttt{make install} as \texttt{root}.
Afterwards it can be controlled by the covert communication channel described in the following chapter \ref{sec:command_and_control}.

\section{TODO: Command and control}\label{sec:command_and_control}
\subsection{Controlling the rootkit}
After it is loaded, the rootkit provides a remote shell to the server specified
in the loading parameters(pk\_ip). The user just has to listen for incoming tcp
connections on port \texttt{20000}, for example "nc -l -p 20000 -vvv".
In order to knock on the port, the user needs to execute 
"python conf\_client\_shell.py" and the rootkit will open a shell.

Commands to the rootkit are encoded in json syntax and sent via a convert tcp
channel to the rootkit. For a full list of the available commands check the next
chapter.
Some commands have boolean arguments and some take arrays as arguments,
respecting the json format. We advise the user to be careful when craftin
control syntax in order to avoid errors.

The following table \ref{tab:commands} lists all commands with a short description of their functionality.

\begin{table}
%\centering
\begin{tabular}{ |l|l| }
\hline
\multicolumn{2}{ |c| }{Process hiding} \\
\hline
\texttt{"hide\_processes":   ["<pid>", ...];} & Hides a specific
process \texttt{<pid>}.\\
\texttt{"unhide\_processes":   ["<pid>", ...];} & Reveals a previously hidden process. \\ \hline

\multicolumn{2}{ |c| }{Module hiding} \\
\hline
\texttt{"hide\_module":   "true";} & Hides the current rootkit
module.\\
\texttt{"unhide\_module":   "true";} & Reveals the current
rootkit module. \\ \hline

\end{tabular}
\caption{Commands to control the rootkit}
\label{tab:commands}
\end{table}

\section{TODO: Implementation}\label{sec:implementation}

\subsection{High-level design}
\subsection{Obtaining address of kernel symbols}
\subsection{System call hooking}\label{sec:implementation_system_call_hooking}
\subsection{File hiding}
\subsection{Process Hiding}
\subsection{Module hiding}
\subsection{Keylogging}
\subsection{Command and control}
\subsection{Privilege Escalation}
\subsection{Socket hiding}
\subsection{Packet hiding}
\subsection{Port knocking}
\subsection{Remote shell}

\section{TODO: Vulnerabilities}\label{sec:vulnerabilities}
\section{TODO: Conclusion}
\end{document}
